\title{Math 239 Fall 2023 Tutorial Questions Week 2}

\date{2023 Sep. 21/22}
\maketitle

\begin{enumerate}
    \question{Even and Odd} For any integer $n \geq 0$, determine the number of compositions of $n$ 
    with $3$ parts where exactly $2$ parts are even.  You need to define 
    a relevant set, a weight function, determine a generating series, 
    and find an explicit formula for the answer.  

    
    \question{Power Series} Find the power series generating functions for each of the following sequences in closed form. Each sequence is defined for all $n \ge 0$.
    \begin{itemize}
        %\item $a_n = n^2$
        \item $a_n = 3^n$
        \item $a_n = 3 \cdot 7^n - 5 \cdot 4^n$
    \end{itemize}
    % let me know if this question is reasonable, I feel like the first bullet might not be since it uses derivatives which I haven't seen in the course notes so I commented it out
    % Who is "me"? In any case I think it would be cool to do it in reverse, ie give the generating function that produces $n^2$ and tell them to show it gives $n^2$. I have written this up below. Also I think that doing this way would be a bit much since most often people do not cover derivatives. -- Alex
    % I have talked to instructor and they said that they will not cover derivatives in class and we should not put them on the tutorial at risk of confusion. --Alex
    % I got excited about it and wrote it up as a bonus question anyway, which I'll put as question 5 with a lot of warning symbols around it. I also okayed this with a prof. -- Alex

   
    
    \question{Some Less-Usual Coefficient Extraction} Let $a,k,l$ be constants. Find the following coefficients.
    \begin{enumerate}
        \item $[x^n] \left[ \frac{1}{2-x^2} \right]$.
        \item $[x^n] \left[ \frac{1}{1-x} \cdot \frac{1}{1-3x} \right]$.
        \item $[x^n] \left[ (1+ax)^k \right]$.
        \item $[x^n] \left[ (1+x^l)^k \right]$.
    \end{enumerate}
    For the last one you might need to put on your Math 135 hat!
    \question{Sum Lemma} Let $\mathscr{S} = \{1, 2, 3, 4, 5, 6\}^4$ be the set of outcomes when rolling four six-sided dice. For $(a, b, c, d) \in \mathscr{S}$, define its weight to be
$\omega(a, b, c, d) = a + b + c + d$. Consider the generating series  $\Phi_\mathscr{S}(x)$ of $\mathscr{S}$ with respect to $\omega$.
\begin{enumerate}
    \item Explain why $\Phi_\mathscr{S}(x) = \left(\frac{x-x^7}{1-x}\right)^4$
    \item How many outcomes in $\mathscr{S}$ have weight 19?
    \item Let $m,d,k$ be positive integers. When rolling $m$ dice, each of which has exactly $d$ sides (numbered with $1,2, \dots, d$ pips, respectively), how many different ways are there to roll a total of $k$ pips on the top faces of the dice? (Part (b) is the case $m=4, d=6, k = 19$.)
\end{enumerate}


    % This following question is from Alex
    \question{Bonus Material: Formal Derivatives} This is not $239$ material and will not be covered in $239$. You should not use this notation or train of thought for assignments or tests.
    
    Consider a generating function
    \begin{align*}
        A(x) = \sum_{i=0}^\infty a_i x^i = a_0 + a_1 x + a_2 x^2 + a_3 x^3 + \cdots.
    \end{align*}
    We define the \textit{formal derivative} of $A(x)$ to be
    \begin{align*}
        \frac{\dif}{ \dif x} A(x) = \sum_{i = 1}^\infty i a_i x^{i-1} = 1 a_1 + 2 a_2 x + 3 a_3 x^2 + 4 a_4 x^3 + \cdots.
    \end{align*}
    \begin{enumerate}
        \item Find $[x^n] \left[ \frac{\dif}{\dif x} A(x) \right]$.
        \item Find $[x^n] \left[ x \frac{\dif }{\dif x} A(x)  \right]$.
        \item Using the \textit{formal derivative} as a ``normal derivative", ie treating the derivative like you would in a first calculus course, find the generating function for the following sequences.
        \begin{enumerate}
            \item $a_n = n$.
            \item $a_n = n^2$.
            \item $a_n = n^k$ for any fixed $k \geq 1$.
        \end{enumerate}
    \end{enumerate}
    \begin{remark}
        The last one might be hard/impossible, but it turns out if you care only about the \textit{asymptotic behaviour} of $a_n$, that is you want $a_n$ to be approximately $n^k$ for large enough $n$, it is much easier. A result from \textit{analytic combinatorics} says that one example of a generating function for a sequence like this is $\frac{k!}{(1-x)^{k+1}}$ (in fact by the math software Sagemath, then $[x^{1000}] \frac{4!}{(1-x)^5} \approx 1.01004513028955 \cdot (1000)^4$, which shows that at $n=1000$ there is only about a $1 \%$ error).
    \end{remark}
\end{enumerate}